\part{Prefacios, Repaso y otras consideraciones}



\chapter{Operaciones sobre los números reales}

\minitoc

\newpage
 
\section*{Introducción} 
\addcontentsline{toc}{section}{Introducción}%para a�adir al �ndice la secci�n no numerada 

Los distintos conjuntos de números surgen de la necesidad de resolver distintas ecuaciones, es decir, a medida que necesitamos resolver ecuaciones más complejas, más se amplían el campo de números con los que podemos actuar:
 
\section{Estructura de los números reales}
Los números reales tiene estructura de cuerpo y te preguntarás ¿ Qué es un cuerpo?
\begin{defi}
Un cuerpo es una terna $(\mathbb{K},+,\cdot)$ donde:
\begin{enumerate}
\item $\mathbb{K}$ es un conjunto de elementos 
\item $+$ es una operación sobre los elementos de $\mathbb{K}$ que cumple:
\begin{itemize}
\item Es una operación \textbf{conmutativa}, es decir, sean $a,b\in \mathbb{K}$ entonces tendremos que $a+b=b+a$
\item Es una operación \textbf{asociativa}, es decir dados $a,b,c \in \mathbb{K}$ tenemos que $a+(b+c)=(a+b)+c$
\item Existe un \textbf{elemento neutro}, es decir $\exists e / e+a=a+e=a$ $\forall a \in \mathbb{K}$.
\item Cada elemento $a \in \mathbb{K} $ existe un elemento \textbf{inverso} que se denota por $a^{-1} $ de tal manera que $a+a^{-1}=a^{-1}+a=e$ (\emph{Esto también se da cuando no se cumple la conmutativa})
\end{itemize}

\item $\cdot$ es una operación que cumple lo siguiente 
\begin{itemize}

\item Es una operación \textbf{asociativa}, es decir dados $a,b,c \in \mathbb{K}$ tenemos que $a\cdot(b\cdot c)=(a\cdot b)\cdot c$

\item Existe un \textbf{elemento neutro} para esta operación  $\exists e / e\cdot a=a \cdot e=a$ $\forall a \in \mathbb{K}$.

\item Para todo elemento $a \in \mathbb{K}$ entonces $\exists a^{-1} /  a \cdot a^{-1}=a^{-1}\cdot a=e $ (\emph{Esto es lo que distingue un cuerpo a un anillo})

\item $\cdot$ es \textbf{distributivo} respecto de $+$ es decir, $a\cdot (b+c)=a \cdot b + a \cdot b$

\end{itemize}
\end{enumerate}

  
\end{defi}
\noindent
\textbf{\emph{Aclaración 1: }} Aunque se denoten como $+, \cdot$ no tenemos por qué usar las definiciones habituales de la suma y la multiplicación. Por ejemplo, la suma y producto de números reales no son iguales que las mismas operaciones para las matrices \emph{(quedaros con ese nombre.) } \\
\textbf{\emph{Aclaración 2: }}  De esta manera que tenemos que lo que llamamos en los números reales la resta es la suma por el inverso y la división es el producto por el inverso.

\begin{ejerci}

Demostrar que $\mathbb{R}$ y $\mathbb{C}$ son cuerpos 

\end{ejerci}



\section{Potencias y sus propiedades}
\begin{defi}
Podemos definir la 
\end{defi}

\chapter{Un segundo cap\'itulo}
%
\minitoc% Crea mini-�ndice al principio del capitulo

\section{}
\section{}


\appendix
\chapter{Mi primer ap\'endice}