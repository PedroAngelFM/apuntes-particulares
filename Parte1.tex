\part{Prefacios, Repaso y otras consideraciones}



\chapter{Operaciones sobre los números reales}

\minitoc

 
\section*{Introducción} 
\addcontentsline{toc}{section}{Introducción}%para a�adir al �ndice la secci�n no numerada 

Los distintos conjuntos de números surgen de la necesidad de resolver distintas ecuaciones, es decir, a medida que necesitamos resolver ecuaciones más complejas, más se amplían el campo de números con los que podemos actuar:
 
\section{Estructura de los números reales}
Los números reales tiene estructura de cuerpo y te preguntarás ¿ Qué es un cuerpo?
\begin{defi}
Un cuerpo es una terna $(K,+,\cdot)$ donde:
\begin{enumerate}
\item $K$ es un conjunto de elementos 
\item $+$ es una operación sobre los elementos de $K$ que cumple:
\begin{itemize}
\item Es una operación conmutativa, es decir, sean $a,b\in K$ entonces tendremos que $a+b=b+a$
\item Es una operación asociativa, es decir dados $a,b,c \in K$ tenemos que $a+(b+c)=(a+b)+c$
\item Existe un elemento neutro, es decir $\exists e / e+a=a+e=a$ $\forall a \in K$ $\exists e /$.
\item
\end{itemize}


\item
\end{enumerate}

\emph{Aclaración: } Aunque se denoten como $+, \cdot$ no tenemos por qué usar las definiciones habituales de la suma y la multiplicación. Por ejemplo, la suma y producto de números reales no son iguales que las mismas operaciones para las matrices \emph{(quedaros con ese nombre.) } 
\end{defi}



\section{Segunda Secci\'on}
La segunda 

\chapter{Un segundo cap\'itulo}
%
\minitoc% Crea mini-�ndice al principio del capitulo

\section{}
\section{}


\appendix
\chapter{Mi primer ap\'endice}