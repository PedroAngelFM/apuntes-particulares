\part{Prefacios, Repaso y otras consideraciones}



\chapter{Operaciones sobre los números reales}

\minitoc

 
\section*{Introducción} 
\addcontentsline{toc}{section}{Introducción}%para a�adir al �ndice la secci�n no numerada 

Los distintos conjuntos de números surgen de la necesidad de resolver distintas ecuaciones, es decir, a medida que necesitamos resolver ecuaciones más complejas, más se amplían el campo de números con los que podemos actuar:
 
\section{Estructura de los números reales}
Los números reales tiene estructura de cuerpo y te preguntarás ¿ Qué es un cuerpo?
\begin{defi}
Un cuerpo es una terna $(K,+,\cdot)$ donde:
\begin{enumerate}
\item $K$ es un conjunto de elementos 
\item
\item
\end{enumerate}
\end{defi}
\input{prueba2.tex}


\section{Segunda Secci\'on}
La segunda 

\chapter{Un segundo cap\'itulo}
%
\minitoc% Crea mini-�ndice al principio del capitulo

\section{}
\section{}


\appendix
\chapter{Mi primer ap\'endice}