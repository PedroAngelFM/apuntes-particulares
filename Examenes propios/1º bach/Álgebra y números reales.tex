\documentclass[11pt,a4paper]{article}
\usepackage[utf8]{inputenc}
\usepackage[spanish]{babel}
\usepackage{amsmath}
\usepackage{amsfonts}
\usepackage{amssymb}
\usepackage{makeidx}
\usepackage{graphicx}
\usepackage[left=2cm,right=2cm,top=2cm,bottom=2cm]{geometry}
\title{Examen álgebra y números complejos}
\author{Pedro Ángel Fraile Manzano }
\begin{document}
\maketitle
\section*{Aclaraciones}
\begin{itemize}
\item Todos los ejercicios deben acompañarse de su razonamiento, todo ejercicio que muestre un resultado sin detallar razonamiento o proceso no será corregido
\item Todos los ejercicios tienen la misma puntuación, en caso de que un ejercicio tenga varios apartados se puntuarán igual a no ser que se diga lo contrario. 
\end{itemize}

\section{Descomposición de polinomios y fracciones algebraicas}
\subsection{Factoriza los siguientes polinomios}
\begin{enumerate}
\item
\item
\item
\item
\end{enumerate}
\subsection{Opera y simplifica las siguientes fracciones algebraicas}
\section{Números complejos $\mathbb{C}$}
\subsection{Calcula usando el binomio de Newton y la forma polar, las siguientes potencias}
\begin{itemize}
\item $(2-2\sqrt{3}i)^5$
\item $(1+3i)^4$
\item $(3+4i)^10$
\item $()$
\end{itemize}
\subsection{Resuelve las siguientes ecuaciones, expresando todas sus raíces, incluidas las complejas}
\end{document}