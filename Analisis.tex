\part{Análisis Matemático}
\chapter{Las sucesiones sobre $\mathbb{R}$}
\minitoc

\newpage
\chapter{Cálculo de Límites}
\minitoc

\newpage
\chapter{Las funciones sobre $\mathbb{R}$}
\minitoc

\newpage
\chapter{Derivabilidad sobre $\mathbb{R}$}

\minitoc


\newpage

\section{Concepto de la derivada}
Para empezar, tenemos que refrescar un concepto de geometría análitica, \emph{la pendiente de una recta}
\begin{defi}
La pendiente de una recta en $\mathbb{R}^2$ \emph{(El plano real)} se define como la cantidad de unidades que avanza la $y$ por cada unidad que avanza la $x$. Es decir, definiendo el incremento de y como $y_1-y_0=\Delta y$ donde $y_1$ es la coordenada y del punto final y $y_0$ lo mismo pero del punto inicial. definimos de manera igual el $\Delta x$. Entonces definimos de manera matemática la fórmula de la pendiente como :
\end{defi}
\begin{equation*}
m=\dfrac{\Delta y}{\Delta x}
\end{equation*}

\noindent
Ahora bien, sea $f(x)$ una función de manera que $f:\mathbb{R}\longrightarrow \mathbb{R}$ de la cual queremos obtener la recta secante que pasa por unos determinados puntos $p_1=(x_1,y_1),p_2=(x_2,y_2)$. Entonces tendremos la siguiente gráfica:\\
\noindent
Tendremos entonces que la fórmula de la recta secante a la función que pasa por esos dos puntos $p_1,p_2$ es la siguiente:
\begin{equation*}
(y-f(x_1))=\dfrac{\Delta f(x)}{\Delta x}(x-x_1)
\end{equation*}
\begin{defi}
A la pendiente de la recta secante a la función $f(x)$ en los puntos $x_1,x_2$ se le conoce como \textbf{\emph{Tasa de Variación Media}}
\end{defi}

\noindent
Supongamos ahora que escribimos $x_1=x$ y $x_2=x+h$ donde $h\in \mathbb{R}$ entonces la ecuación anterior queda como:
\begin{equation*}
(y-f(x_1))=\dfrac{f(x_1+h)-f(x_1)}{h}(x-x_1)
\end{equation*}

\noindent
Si después de esto, si hacemos que la $h\to 0$ obtendremos la recta tangente de manera que la pendiente  $\displaystyle \lim_{h\rightarrow 0}\dfrac{f(x+h)-f(x)}{h}$.

\noindent
Es ese límite lo que definimos como \emph{Derivada de una función}.
\begin{defi}
Llamaremos derivada de $f(x)$ en el punto $a$ al límite
\begin{equation*}
\lim_{h\to 0}\dfrac{f(a+h)-f(a)}{h}
\end{equation*}

\end{defi}
\newpage
\section{Derivabilidad de una función}
\begin{defi}
Diremos que una función es derivable en $a$ si existe el límite $\displaystyle \lim_{h\rightarrow 0}\dfrac{f(x+h)-f(x)}{h}$.
\end{defi}
\begin{defi}
Diremos que una función es derivable si lo es en todos los puntos del dominio.
\end{defi}
\subsection{Estudio de la derivabilidad de una función}

\newpage
\section{Tabla de derivadas}

Para empezar hay que tener en cuenta estas derivadas de operaciones de funciones básicas, sumar y restar, producto y división, producto por un escalar y composición

\noindent
Sean $a \in \mathbb{R}$ 

\section{Algunas demostraciones de fórmulas de derivadas}
\newpage

\chapter{Aplicaciones de la derivada}
\minitoc

\newpage
\section{Cálculo de mínimos y máximos}
\section{Cálculo de la curvatura de las funciones}
\section{Optimización de funciones}
\chapter{Representación de funciones}
\minitoc

\newpage
\section{Dominio}
\section{Continuidad}
\section{Corte con los ejes}
\section{Simetría}
\section{Asíntotas}
\section{Monotonía}
\section{Curvatura}
\chapter{Integración sobre $\mathbb{R}$}
\minitoc