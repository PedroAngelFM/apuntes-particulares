\part{Análisis Matemático}
\chapter{Preámbulos para análisis}
\minitoc

\newpage
\section{Topología sobre $\mathbb{R}$}
La topología ocupa un lugar muy destacado cuando se trata del análisis matemático ya que es la rama que estudia que ocurre con ciertas propiedades de proximidad cuando a un conjunto le aplicas lo que hemos llamado funciones continuas en anteriores cursos. 

\begin{defi}
Una \textbf{topología}, $\tau$, sobre el conjunto $A$  es una familia de subconjuntos de  finita o no que cumple las siguientes características:
\begin{itemize}
\item $A$ y $\emptyset$ pertenecen a $\tau$.
\item Dado $\lbrace A_i \rbrace_{i\in I}$ una familia arbitraria (puede ser finita o no) de elementos de $\tau$ entonces $\cup_{i\in I} A_i$ también pertenece a la topología.
\item Dado $\lbrace A_i \rbrace_{i=0}^{i=k}$ una familia finita de elementos de la topología entonces $\cap_{i=0}^{i=k} A_i$ es también un elemento de la topología.
\end{itemize}

A esta forma de definir una topología lo llamamos definir una topología por \emph{abiertos}, ya que a los elementos de la topología definida así se les llama conjuntos abiertos. 
\end{defi}
\begin{ejercicio}
Comprobar que $\mathbb{R}$ con los intervalos abiertos es una topología. 
\end{ejercicio}

\begin{defi}
Diremos que un conjunto es \textbf{cerrado} cuando su complementario sea abierto. 
\end{defi}

\paragraph*{Aclaración: } Un conjunto puede no ser ni abierto ni cerrado y puede ser los dos a la vez, como por ejemplo el $\emptyset$ ya que su contrario que sería el total es abierto, pero como está en la topología $\tau$, entonces es abierto. Ambas cosas no son contradictorias. 

Ahora bien, supongamos que ya tenemos estas cualidades de espacio topológico, ahora vamos a definir lo que es un entorno abierto de un punto a. 
\begin{defi}
Un entorno no es más que la vecindad de un punto \emph{(Los puntos cercanos a él)}, y diremos que un entorno $E$ es abierto si $\forall x \in E $ existe un abierto que está contenido en el entorno. 

\end{defi}

Ahora nos falta definir una medida sobre $\mathbb{R}$ que nos permita decir de manera clara y concisa lo que está y lo que no está en las vecindades del punto. La medida más habitual sobre $\mathbb{R}$ es el valor absoluto. 





\chapter{Cálculo de Límites}
\minitoc

\newpage
\chapter{Las funciones sobre $\mathbb{R}$}
\minitoc

\newpage
\chapter{Derivabilidad sobre $\mathbb{R}$}

\minitoc


\newpage

\section{Concepto de la derivada}
Para empezar, tenemos que refrescar un concepto de geometría análitica, \emph{la pendiente de una recta}
\begin{defi}
La pendiente de una recta en $\mathbb{R}^2$ \emph{(El plano real)} se define como la cantidad de unidades que avanza la $y$ por cada unidad que avanza la $x$. Es decir, definiendo el incremento de y como $y_1-y_0=\Delta y$ donde $y_1$ es la coordenada y del punto final y $y_0$ lo mismo pero del punto inicial. definimos de manera igual el $\Delta x$. Entonces definimos de manera matemática la fórmula de la pendiente como :
\end{defi}
\begin{equation*}
m=\dfrac{\Delta y}{\Delta x}
\end{equation*}

\noindent
Ahora bien, sea $f(x)$ una función de manera que $f:\mathbb{R}\longrightarrow \mathbb{R}$ de la cual queremos obtener la recta secante que pasa por unos determinados puntos $p_1=(x_1,y_1),p_2=(x_2,y_2)$. Entonces tendremos la siguiente gráfica:\\
\noindent
Tendremos entonces que la fórmula de la recta secante a la función que pasa por esos dos puntos $p_1,p_2$ es la siguiente:
\begin{equation*}
(y-f(x_1))=\dfrac{\Delta f(x)}{\Delta x}(x-x_1)
\end{equation*}
\begin{defi}
A la pendiente de la recta secante a la función $f(x)$ en los puntos $x_1,x_2$ se le conoce como \textbf{\emph{Tasa de Variación Media}}
\end{defi}

\noindent
Supongamos ahora que escribimos $x_1=x$ y $x_2=x+h$ donde $h\in \mathbb{R}$ entonces la ecuación anterior queda como:
\begin{equation*}
(y-f(x_1))=\dfrac{f(x_1+h)-f(x_1)}{h}(x-x_1)
\end{equation*}

\noindent
Si después de esto, si hacemos que la $h\to 0$ obtendremos la recta tangente de manera que la pendiente  $\displaystyle \lim_{h\rightarrow 0}\dfrac{f(x+h)-f(x)}{h}$.

\noindent
Es ese límite lo que definimos como \emph{Derivada de una función}.
\begin{defi}
Llamaremos derivada de $f(x)$ en el punto $a$ al límite
\begin{equation*}
\lim_{h\to 0}\dfrac{f(a+h)-f(a)}{h}
\end{equation*}

\end{defi}
\newpage
\section{Derivabilidad de una función}
\begin{defi}
Diremos que una función es derivable en $a$ si existe el límite $\displaystyle \lim_{h\rightarrow 0}\dfrac{f(x+h)-f(x)}{h}$.
\end{defi}
\begin{defi}
Diremos que una función es derivable si lo es en todos los puntos del dominio.
\end{defi}
\subsection{Estudio de la derivabilidad de una función}

\newpage
\section{Tabla de derivadas}

Para empezar hay que tener en cuenta estas derivadas de operaciones de funciones básicas, sumar y restar, producto y división, producto por un escalar y composición

\noindent
Sean $a \in \mathbb{R}$ 

\section{Algunas demostraciones de fórmulas de derivadas}
\newpage

\chapter{Aplicaciones de la derivada}
\minitoc

\newpage
\section{Cálculo de mínimos y máximos}
\section{Cálculo de la curvatura de las funciones}
\section{Optimización de funciones}
\chapter{Representación de funciones}
\minitoc

\newpage
\section{Dominio}
Recordemos lo que era el dominio de una función de manera precisa
\begin{defi}
El dominio de una función o aplicación $f: X\longrightarrow Y$ es el subconjunto de puntos $x \in X$ para el cual existe $f(x)$. 
\end{defi}
\noindent
Para que lo entendamos de nuevo la función $f$ en el caso que nos ocupa es algo a lo que le entran números, la función hace una operación, \emph{(En el caso de $2x$ sería que lo multiplica por dos)} y te devuelve ese valor operado. 
\noindent
El dominio son los números que la función puede operar sin  romperse, es decir, sin hacer cosas raras que no se pueden hacer en los números reales como por ejemplo:
\begin{itemize}
\item Dividir entre 0: Por tanto si tenemos una fracción habrá que comprobar cuando el denominador se hace 0. (Plantear la ecuación)
\item Hacer la raíz de un número negativo: Para poder hacerlo tendríamos que extender nuestro campo a $\mathbb{C}$ lo cual ahora mismo se nos escapa de nuestro alcance. Se plantea la inecuación
\item Hacer el logaritmo de un número negativo 
\end{itemize}

\subsection*{Dominios de las funciones elementales}
\noindent
Ahora vamos a ir desmenuzando los tipos de funciones que conocemos y analizando su dominio:
\begin{itemize}
\item \textbf{Funciones Polinómicas} Son las funciones del tipo $a_n\cdot x^n+\ldots+ a_o$ en este caso, su dominio es todo $\mathbb{R}$.
\item \textbf{Funciones Racionales} Son las funciones del tipo $\displaystyle \dfrac{a_n\cdot x^n+\ldots+ a_0}{b_m\cdot x^m+\ldots+ b_0}$ en este caso, su dominio es todo $\mathbb{R}$, salvo los puntos en los que se anule el denominador, por o tanto hay que resolver la ecuación $b_m\cdot x^m+\ldots+ b_0=0$.
\item \textbf{Funciones Irracionales} Son las de tipo $f(x)=\sqrt[n]{g(x)}$ tenemos que el dominio de $f(x)$ es:
\begin{itemize}
\item El mismo que $g(x)$ si n es \emph{impar}
\item Si n es par el dominio de $f(x)$ son los x que cumplen que  $g(x)\geq 0$. 
\end{itemize}
\item \textbf{Funciones Exponenciales}
Son de la forma $f(x)=a^{g(x)}$ con $a>0$ y $a\neq 1$, su dominio es $\mathbb{R}$.
\item \textbf{Funciones Logarítmicas}
Son de la forma $f(x)=log_a g(x)$ $a>0$ y  su dominio son los $x$ tales que $g(x)>0$.
\item \textbf{Funciones Trigonométricas Circulares} 
Tanto $f(x)=cos(x)$ como $f(x)=sen(x)$ su dominio es $\mathbb{R}$ y de aquí se pueden considerar el resto de funciones trigonométricas. 
\end{itemize}
\section{Simetría y periodicidad}
\noindent
La simetría de una función se define de la siguiente manera:

\begin{defi}
Una función $f:A \longrightarrow \mathbb{R}$ es par si $\forall x \in A$ se cumple que $f(-x)=f(x)$. Es decir, es simétrica respecto del eje de coordenadas $OY$
\end{defi}
%%Aquí va una imagen
\begin{defi}
Una función $f:A \longrightarrow \mathbb{R}$ es impar si $\forall x \in A$ se cumple que $f(-x)=f(x)$. Es decir, es simétrica respecto del origen de coordenadas 
\end{defi}
%Aquí va una imagen%
\noindent
Estas cualidades lo que nos permiten es reducir el tamaño del conjunto de puntos a estudiar, en el caso de la simetría nos permite reducir a la parte positiva de los reales y la periodicidad a un solo periodo. 

\section{Continuidad}
\section{Corte con los ejes}
\section{Asíntotas}
\section{Monotonía}
\section{Curvatura}
\chapter{Integración sobre $\mathbb{R}$}
\minitoc