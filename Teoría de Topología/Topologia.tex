\documentclass[12pt,a4paper]{book}

\usepackage{fancyhdr}
\usepackage{amssymb}
\usepackage{mathtools}

\usepackage[spanish]{babel}

\usepackage[spanish]{minitoc}
\usepackage{minitoc}
\usepackage{amsthm}
\theoremstyle{definition}
\newtheorem{defi}{Definición}[chapter]
\theoremstyle{definition}
\newtheorem*{ejerci}{Ejercicio Propuesto}
\theoremstyle{definition}
\newtheorem{ejercicio}[subsection]{Ejercicio}
\newtheorem*{solution}{Sol:}
\newtheorem{lemma}{Lema}[chapter]
\newtheorem{proposition}{Proposición}[chapter]
\theoremstyle{lemma}


\addto{\captionsspanish}{\renewcommand{\mtctitle}{\'Indice del cap\'itulo}}

\renewcommand{\mtctitle}{\'Indice}% as� tambi�n cambiamos la cabecera del los minitoc


\usepackage{graphicx}

\usepackage{hyperref} % url, label, citas, tableofcontents

\title{Teoría de Topología}
\author{Pedro Ángel Fraile Manzano} 
\date{\today}
\begin{document}
%%%%%%%%%%%%%%%
%\input{Portada_book}%Portada con entorno titlepage
%\input{book_cabecera}%Personalizamos la portada a mano
\maketitle%Aparece el t�tulo (\title{} \author{} \date{\today)}), es distinto a la portada
%%%Personalizando el �ndice general (toc)
\addtocontents{toc}{\hspace{-7.5mm} \textbf{Apartados del libro}}
\addtocontents{toc}{\hfill \textbf{P\'agina} \par}
\addtocontents{toc}{\vspace{-2mm} \hspace{-7.5mm} \hrule \par}
\dominitoc%para crear �ndices en cada cap�tulo. Requiere paquete {minitoc
\renewcommand{\contentsname}{Contenidos Generales}%Nombre del �ndice general
\tableofcontents
\adjustmtc{} %indice en la intro
\chapter{Espacios Métricos}

\begin{defi}
Si E es un $\mathbb{R}$ espacios vectorial una norma sobre E es una aplicación $\parallel\parallel\colon E \Longrightarrow \mathbb{R}$ que para cada $e,e'\in E$ y $a\in \mathbb{R}$ que verifica:
\begin{enumerate}
\item $\parallel e \parallel \geq 0; \parallel e\parallel=0 \Leftrightarrow e=0$
\item Desigualdad triangular $\parallel e +e'\parallel\leq \parallel e \parallel +\parallel e' \parallel$
\item $\parallel ae \parallel=a\parallel e \parallel$
\end{enumerate}
Si E es un espacio vectorial y $\parallel \parallel$ una norma como la definida antes, el par $(E,\parallel \parallel)$ recibe el nombre de espacio normado 
\end{defi}

\begin{defi}
Sea E un $\mathbb{R}$-espacio vectorial. Diremos que un producto escalar $$T_2\colon E\times E \Longrightarrow \mathbb{R}$$ es euclídeo si es definido positivo, es decir si para cada $e \in E$, $T_2(e,e)\geq 0$ y $T_2(e,e)=0\Leftrightarrow e=0$. Un espacio euclídeo es un par $(E,T_2)$ donde 
\end{defi}

\begin{lemma}
Sea $(E,T_2)$ un espacio euclídeo y $\parallel e \parallel=sqrt{T_2(e,e)}$ Dados $e,e' \in E$ se verifica que:
\begin{enumerate}
\item \emph{Desigualdad de Schwarz:} $|T_2(e,e')|\leq \parallel e \parallel \cdot \parallel e' \parallel$
\item \emph{Desigualdad de Minkowski:}$\parallel e+e' \parallel \leq \parallel e\parallel +\parallel e' \parallel$

\end{enumerate}
\end{lemma}

\noindent
\begin{proof}
Vamos a separar cada una de las proposiciones:
\begin{enumerate}
\item Para cada $a \in \mathbb{R}$ se tiene que: 
\begin{equation*}
0\leq T_2(e+ae',e+ae')=T_2(e,e)+2aT_2(e,e')+a^2T_2(e',e')
\end{equation*}
Esta expresión se da ya que el $T_2$ es una aplicación bilineal, ya que es una aplicación la cual es lineal en cada uno de los términos. En este caso, el último término de la igualdad es un polinomio de grado 2 que no toma valores negativos, ya que ninguno de los productos euclídeos son positivos.Entonces, como no tiene solución el discriminante de este polinomio no puede ser positivo
\begin{equation*}
4 T_2(e,e')-4T_2(e,e)T_2(e',e')\leq 0
\end{equation*}

Por tanto, $T_2(e,e')^2\leq T_2(e,e)T_2(e',e')$ por tanto, se concluye. 
\item Por darse la anterior proposición tenemos que
\begin{align*}
\parallel e+e' \parallel^2 &= T_2(e+e',e+e')=\\
=T_2(e,e)+2T_2(e,e')+T_2(e',e')&\leq \parallel e\parallel^2+\parallel e\parallel\parallel e'\parallel+\parallel e'\parallel^2=(\parallel e\parallel+\parallel e'\parallel)^2
\end{align*}
y tomando las raíces cuadradas se concluye
\end{enumerate}
\end{proof}

\begin{proposition}
Si $(E,T_2)$ es un espacio euclídeo, la aplicación $\parallel\parallel\colon E \longrightarrow \mathbb{R}$ definida como $\parallel e\parallel= \sqrt{T_2(e,e)}$ es una norma en $E$.
\end{proposition}

\noindent
La norma asociada con un producto escalar se denota habitualmente como $\parallel\parallel_2$, en $\mathbb{R}^n$ la norma $\parallel\parallel_2$ se puede calcular como:
$$
\parallel x\parallel_2=\sqrt{\sum_{i=1}^n x_i^2}
$$
\noindent
Esta será la norma que no se especifique otra cosa 

\begin{defi}
Sea $X$ un copnjunto, una distancia en $X$ es una aplicación $d\colon X\times X\longrightarrow \mathbb{R}^+$ para $x,y,z\in X$ se verifica:
\begin{itemize}
\item $d(x,y)\geq 0 ; d(x,y)=0 \Leftrightarrow x=y$
\item $d(x,y)=d(y,x)$
\item Desigualdad triangular $d(x,z)\leq d(x,y)+d(y,z)$
\end{itemize}
\end{defi}

Las nociones anteriores únicamente se pueden aplicar a espacios vectoriales, pero la de distancia se puede aplicar a cualquier conjunto $X$.

\chapter{Espacios Topológicos}
\begin{defi}
Sea $(X,d)$ un espacio métrico, dados $x\in X$ $r\in \mathbb{R}^+$, llamaremos bola abierta de centro x y radio r al conjunto 
$$
B(x,r)=\lbrace y\in X \colon d(x,y)<r\rbrace
$$
y la bola cerrada de centro x y radio r.
$$
\overline{B}(x,r)=\lbrace y\in X \colon d(x,y)\leq r\rbrace
$$
\end{defi}

Dependiendo de la distancia y la norma que se haya definido, las bolas tienen distinta forma, por ejemplo la norma infinito una bola es un cuadrado. 

\begin{defi}
Sea $(X,d)$ un espacio métrico. Diremos que un subconjunto U de X es abierto si para cada $x\in U \exists r>0\colon B(x,r)\subseteq U$
\end{defi}
En un espacio métrico las bolas abiertas son abiertos de la topología 

\begin{proposition}
Los abiertos de un espacio métrico (X,d), verifican las siguientes propiedades: 
\begin{enumerate}
\item $\emptyset$ y $X$ son abiertos
\item Si $U$ y  $V$ son abiertos entonces $U \cap V$ también lo es. (Se cumple para cualquier conjunto finito de conjuntos)
\item Si $\lbrace U_\alpha \rbrace_{\alpha\in A}$ es una familia de abiertos, entonces $\bigcup_{\alpha\in A} U_\alpha$ también es abierto. 
\end{enumerate}
\end{proposition}

\end{document}