\chapter{Espacios Topológicos}
\begin{defi}
Sea $(X,d)$ un espacio métrico, dados $x\in X$ $r\in \mathbb{R}^+$, llamaremos bola abierta de centro x y radio r al conjunto 
$$
B(x,r)=\lbrace y\in X \colon d(x,y)<r\rbrace
$$
y la bola cerrada de centro x y radio r.
$$
\overline{B}(x,r)=\lbrace y\in X \colon d(x,y)\leq r\rbrace
$$
\end{defi}

Dependiendo de la distancia y la norma que se haya definido, las bolas tienen distinta forma, por ejemplo la norma infinito una bola es un cuadrado. 

\begin{defi}
Sea $(X,d)$ un espacio métrico. Diremos que un subconjunto U de X es abierto si para cada $x\in U \exists r>0\colon B(x,r)\subseteq U$
\end{defi}
En un espacio métrico las bolas abiertas son abiertos de la topología 

\begin{proposition}
Los abiertos de un espacio métrico (X,d), verifican las siguientes propiedades: 
\begin{enumerate}
\item $\emptyset$ y $X$ son abiertos
\item Si $U$ y  $V$ son abiertos entonces $U \cap V$ también lo es. (Se cumple para cualquier conjunto finito de conjuntos)
\item Si $\lbrace U_\alpha \rbrace_{\alpha\in A}$ es una familia de abiertos, entonces $\bigcup_{\alpha\in A} U_\alpha$ también es abierto. 
\end{enumerate}
\end{proposition}
