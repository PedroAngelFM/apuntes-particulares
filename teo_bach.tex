\documentclass[12pt,a4paper]{book}

\usepackage{fancyhdr}
\usepackage{amssymb}
\usepackage{mathtools}

\usepackage[spanish]{babel}

\usepackage[spanish]{minitoc}
\usepackage{minitoc}
\usepackage{amsthm}
\theoremstyle{definition}
\newtheorem{defi}{Definición}[section]
\theoremstyle{definition}
\newtheorem*{ejerci}{Ejercicio Propuesto}
\theoremstyle{definition}
\newtheorem{ejercicio}[subsection]{Ejercicio}
\newtheorem*{solution}{Sol:}
\addto{\captionsspanish}{\renewcommand{\mtctitle}{\'Indice del cap\'itulo}}

\renewcommand{\mtctitle}{\'Indice}% as� tambi�n cambiamos la cabecera del los minitoc


\usepackage{graphicx}

\usepackage{hyperref} % url, label, citas, tableofcontents

\title{Teoría de Matemáticas de Bachillerato}
\author{Pedro Ángel Fraile Manzano} 
\date{\today}
\begin{document}
%%%%%%%%%%%%%%%
%\input{Portada_book}%Portada con entorno titlepage
%\input{book_cabecera}%Personalizamos la portada a mano
\maketitle%Aparece el t�tulo (\title{} \author{} \date{\today)}), es distinto a la portada
%%%Personalizando el �ndice general (toc)
\addtocontents{toc}{\hspace{-7.5mm} \textbf{Apartados del libro}}
\addtocontents{toc}{\hfill \textbf{P\'agina} \par}
\addtocontents{toc}{\vspace{-2mm} \hspace{-7.5mm} \hrule \par}
\dominitoc%para crear �ndices en cada cap�tulo. Requiere paquete {minitoc
\renewcommand{\contentsname}{Contenidos Generales}%Nombre del �ndice general
\tableofcontents
\adjustmtc{} %indice en la intro
\part{Prefacios, Repaso y otras consideraciones}



\chapter{Operaciones sobre los números reales}

\minitoc

\newpage
 
\section*{Introducción} 
\addcontentsline{toc}{section}{Introducción}%para a�adir al �ndice la secci�n no numerada 

Los distintos conjuntos de números surgen de la necesidad de resolver distintas ecuaciones, es decir, a medida que necesitamos resolver ecuaciones más complejas, más se amplían el campo de números con los que podemos actuar:
 
\section{Estructura de los números reales}
Los números reales tiene estructura de cuerpo y te preguntarás ¿ Qué es un cuerpo?
\begin{defi}
Un cuerpo es una terna $(\mathbb{K},+,\cdot)$ donde:
\begin{enumerate}
\item $\mathbb{K}$ es un conjunto de elementos 
\item $+$ es una operación sobre los elementos de $\mathbb{K}$ que cumple:
\begin{itemize}
\item Es una operación \textbf{conmutativa}, es decir, sean $a,b\in \mathbb{K}$ entonces tendremos que $a+b=b+a$
\item Es una operación \textbf{asociativa}, es decir dados $a,b,c \in \mathbb{K}$ tenemos que $a+(b+c)=(a+b)+c$
\item Existe un \textbf{elemento neutro}, es decir $\exists e / e+a=a+e=a$ $\forall a \in \mathbb{K}$.
\item Cada elemento $a \in \mathbb{K} $ existe un elemento \textbf{inverso} que se denota por $a^{-1} $ de tal manera que $a+a^{-1}=a^{-1}+a=e$ (\emph{Esto también se da cuando no se cumple la conmutativa})
\end{itemize}

\item $\cdot$ es una operación que cumple lo siguiente 
\begin{itemize}

\item Es una operación \textbf{asociativa}, es decir dados $a,b,c \in \mathbb{K}$ tenemos que $a\cdot(b\cdot c)=(a\cdot b)\cdot c$

\item Existe un \textbf{elemento neutro} para esta operación  $\exists e / e\cdot a=a \cdot e=a$ $\forall a \in \mathbb{K}$.

\item Para todo elemento $a \in \mathbb{K}$ entonces $\exists a^{-1} /  a \cdot a^{-1}=a^{-1}\cdot a=e $ (\emph{Esto es lo que distingue un cuerpo a un anillo})

\item $\cdot$ es \textbf{distributivo} respecto de $+$ es decir, $a\cdot (b+c)=a \cdot b + a \cdot b$

\end{itemize}
\end{enumerate}

  
\end{defi}
\newpage
\noindent
\textbf{\emph{Aclaración 1: }} Aunque se denoten como $+, \cdot$ no tenemos por qué usar las definiciones habituales de la suma y la multiplicación. Por ejemplo, la suma y producto de números reales no son iguales que las mismas operaciones para las matrices \emph{(quedaros con ese nombre.) } \\
\textbf{\emph{Aclaración 2: }}  De esta manera que tenemos que lo que llamamos en los números reales la resta es la suma por el inverso y la división es el producto por el inverso.

\begin{ejerci}

Demostrar que $\mathbb{R}$ y $\mathbb{C}$ son cuerpos 

\end{ejerci}



\section{Potencias y Logaritmos}
\begin{defi}
Podemos definir las potencias como  $a^n=\overbrace{a \cdot\ldots \cdot a}^{\text{n veces}}$. Una vez entendido esto tenemos las siguientes propiedades
\end{defi}

\paragraph{Propiedades}
\begin{enumerate}
\item $a^1=a$ y $a^0=1$ para cualquier $a\in\mathbb{R}$
\item $a^{-1}=\dfrac{1}{a}$
\item $a^n\cdot a^m=a^{n+m}$
\item $\dfrac{a^n}{a^m}=a^{n-m} $
\item $\left( a^n \right) ^m =a^{n \cdot m}$
\item $\sqrt[n]{a}=a^{\frac{1}{n}}$
\item $(a \cdot b)^n=a^n \cdot b^n$
\item $ \left(\dfrac{a}{b}\right)^n=\dfrac{a^n}{b^n}$
\end{enumerate}
\paragraph{Demostración}
\begin{enumerate}
\item Para la primera demostración no hace falta más que decir que estamos ``poniendo'' sólo una a y que $a^0=1$ es básicamente proveniente del álgebra $\mathbb{Z}$ modular.
 
\item En este caso,  tenemos que al utilizar la propiedad 3 quedará más clara pero si nosotros tenemos $a^1 \cdot a^{-1}=a^0=1 \Rightarrow a^{-1}=\dfrac{1}{a}$

\item Ahora tenemos que $a^n\cdot a^m=\overbrace{a \cdot\ldots \cdot a}^{\text{n veces}}\cdot \overbrace{a \cdot\ldots \cdot a}^{\text{m veces}}=\overbrace{a \cdot\ldots \cdot a}^{\text{n+m veces}}=a^{m+n}$
\item Si combinamos la propiedad 2 y 3 queda probado $\dfrac{a^n}{a^m}=a^n\cdot \dfrac{1}{a^m}=a^n \cdot a^{-m}=a^{n-m}$
\item Este se debe a que estamos multiplicando paquetitos del producto de n a's, es decir ,
$ \left( a^n \right) ^m =\overbrace{a^n \cdot \ldots \cdot a^n}^{\text{m veces}}=\overbrace{\underbrace{a \cdot \ldots \cdot a}_{\text{n veces}}\cdot \ldots \cdot \underbrace{a \cdot\ldots \cdot a}_{\text{n veces}}}^{\text{m veces}}=\overbrace{a \cdot\ldots \cdot a}^{\text{n m veces}}=a^{m n}$
\item Haciendo un razonamiento análogo pero con el producto lo tenemos 
\item Tenemos lo siguiente $(a \cdot b)^n=\overbrace{a\cdot b \cdot\ldots \cdot a\cdot b}^{\text{n veces}}=\overbrace{a \cdot\ldots \cdot a}^{\text{n veces}}\cdot \overbrace{b \cdot\ldots \cdot b}^{\text{n veces}}=a^n \cdot b^n$
\item Utilizando un razonamiento similar al anterior lo tenemos cambiando únicamente $b$ por $b^{-1}$
\end{enumerate}

\begin{defi}
Definimos el logaritmo de $b \in \mathbb{R}^+$ en base $a>0$ de la siguiente manera
\begin{equation}
log_a b=x \Leftrightarrow a^x=b
\end{equation}
\end{defi}

Esta definición nos permite "traducir" de logaritmos a potencias y es lo que se utiliza para demostrar las siguientes propiedades
%%@todo Lo de propiedades hay que hacer un entorno no numerado con el estilo de definition y lo mismo para las demostraciones 
\paragraph{Propiedades: }
Sean $P,Q,a \in \mathbb{R}^+ $
\begin{enumerate}
\item $log_a 1=0$
\item $log_a a=1$
\item $log_a (P\cdot Q)=log_a P +log_a Q$
\item $log_a \left(\dfrac{P}{Q}\right)=log_a P -log_a Q$
\item $log_a P^n=n\cdot log_a P$
\end{enumerate}

\begin{ejerci}
Se propone al lector la demostración de estas propiedades utilizando la definición de logaritmos y las propiedades de las potencias.
\end{ejerci}
\newpage
\section{Resolución de ecuaciones exponenciales}

\begin{defi}
Podemos definir una ecuación exponencial como aquella que tiene la incógnita en el exponente
\end{defi}
\newpage
\section{Resolución de ecuaciones logarítmicas}
\begin{defi}
Podemos definir una ecuación exponencial como aquella que tiene la incógnita dentro de un logaritmo.
\end{defi}



\chapter{Polinomios sobre el cuerpo de los reales y ecuaciones}

\minitoc

\section{Conceptos básicos}
\section{Operaciones con polinomios}
\section{Divisibilidad de polinomios}

\chapter{Ecuaciones polinómicas}
\minitoc

\section{Ecuaciones lineales}
\section{Ecuaciones parabólicas}
\section{Ecuaciones de grado mayor que 2}

\chapter{Inecuaciones}
\minitoc

\section{Inecuaciones lineales}
\section{Inecuaciones no lineales}



\part{Análisis Matemático}
\chapter{Preámbulos para análisis}
\minitoc

\newpage
\section{Topología sobre $\mathbb{R}$}
La topología ocupa un lugar muy destacado cuando se trata del análisis matemático ya que es la rama que estudia que ocurre con ciertas propiedades de proximidad cuando a un conjunto le aplicas lo que hemos llamado funciones continuas en anteriores cursos. 

\begin{defi}
Una \textbf{topología}, $\tau$, sobre el conjunto $A$  es una familia de subconjuntos de  finita o no que cumple las siguientes características:
\begin{itemize}
\item $A$ y $\emptyset$ pertenecen a $\tau$.
\item Dado $\lbrace A_i \rbrace_{i\in I}$ una familia arbitraria (puede ser finita o no) de elementos de $\tau$ entonces $\cup_{i\in I} A_i$ también pertenece a la topología.
\item Dado $\lbrace A_i \rbrace_{i=0}^{i=k}$ una familia finita de elementos de la topología entonces $\cap_{i=0}^{i=k} A_i$ es también un elemento de la topología.
\end{itemize}

A esta forma de definir una topología lo llamamos definir una topología por \emph{abiertos}, ya que a los elementos de la topología definida así se les llama conjuntos abiertos. 
\end{defi}
\begin{ejercicio}
Comprobar que $\mathbb{R}$ con los intervalos abiertos es una topología. 
\end{ejercicio}

\begin{defi}
Diremos que un conjunto es \textbf{cerrado} cuando su complementario sea abierto. 
\end{defi}

\paragraph*{Aclaración: } Un conjunto puede no ser ni abierto ni cerrado y puede ser los dos a la vez, como por ejemplo el $\emptyset$ ya que su contrario que sería el total es abierto, pero como está en la topología $\tau$, entonces es abierto. Ambas cosas no son contradictorias. 

Ahora bien, supongamos que ya tenemos estas cualidades de espacio topológico, ahora vamos a definir lo que es un entorno abierto de un punto a. 
\begin{defi}
Un entorno no es más que la vecindad de un punto \emph{(Los puntos cercanos a él)}, y diremos que un entorno $E$ es abierto si $\forall x \in E $ existe un abierto que está contenido en el entorno. 

\end{defi}

Ahora nos falta definir una medida sobre $\mathbb{R}$ que nos permita decir de manera clara y concisa lo que está y lo que no está en las vecindades del punto. La medida más habitual sobre $\mathbb{R}$ es el valor absoluto. 


\begin{defi}
Una medida es una aplicación 
$$
\begin{array}{rccl}
||  \colon & \mathbb{R} \times \mathbb{R} & \longrightarrow & \mathbb{R}\\
&(x,y) & \longmapsto & dist(x,y)
\end{array}
$$
\noindent
Que cumple las siguientes propiedades:
\begin{itemize}
\item $\forall x \in \mathbb{R} $ se cumple que $dist(x,x)=0$
\item $\forall x,y\in \mathbb{R}$ se cumple que $dist(x,y)=dist(y,x)$ 
\item $\forall x,y,z \in \mathbb{R}$ se cumple que $dist(x,z)\leq dist(x,y)+dist(y,z)$ Es decir, el camino más corto es el camino directo entre dos puntos.
\end{itemize}

\end{defi}

\begin{ejercicio}
Probar que el valor absoluto es una medida sobre los números reales. 
\end{ejercicio}

Ahora, vamos a definir la principal definición para la que usamos la topología. Las funciones continuas. 

\begin{defi}
Una aplicación
$$
\begin{array}{rccl}
f  \colon & X & \longrightarrow & Y\\

\end{array}
$$
Diremos que es continua si $\forall E$ abierto de $B$ entonces se cumple que $f^{-1}(B)$ es oto conjunto abierto. 
\end{defi}

Esta definición nos quiere decir que este tipo de funciones lo que hace es mandar puntos que están cerca a puntos que siguen cerca por así decirlo,ya que un entorno de f(a) viene de un entorno de a.


Continuará...


\chapter{Cálculo de Límites}
\minitoc

\newpage

\section{Preámbulo sobre las sucesiones reales}
\begin{defi}
Definimos una sucesión de números reales $\lbrace a_k \rbrace$ como una aplicación de la forma: 
$$
\begin{array}{rccl}
\lbrace a_k \rbrace  \colon & \mathbb{N} & \longrightarrow & \mathbb{R}\\
& k & \longmapsto & a_k
\end{array}
$$
\end{defi}
\noindent
Tenemos que tener claros unos cuantos conceptos sobre sucesiones antes de ponernos a definir lo que es una función o sucesión convergente y que es eso de convergente. 

\begin{defi}
Se dice que una sucesión $\lbrace a_n \rbrace$ es convergente en $\mathbb{R}$, o que es convergente a $l\in \mathbb{R}$ o que su límite es $l$ cuando si $\forall \varepsilon \geq 0$ y $\varepsilon \in \mathbb{R}$ $\exists n_0\in \mathbb{N}$ de manera que cuando $n\leq n_0\Leftarrow |l-a_n|\leq \varepsilon$
\end{defi}

\paragraph*{\textit{Aclaración:}} Una sucesión que sea convergente tiene un único límite.\\[2ex]
\noindent
Esta definición lo que nos dice es que cogiendo un $\varepsilon$ arbitrariamente pequeño se puede encontrar un $n_0 \in \mathbb{N}$ de manera que la distancia entre el límite y un término posterior de $\lbrace a_n \rbrace$ a $a_{n_0}$ es menor que ese $\varepsilon$\\

\noindent
Podemos entonces extender esta definición a una aplicación del tipo
\begin{equation*}
\begin{array}{rccl}
f \colon & \mathbb{R} & \longrightarrow & \mathbb{R}\\

\end{array}
\end{equation*}

Entonces dejemos definido esta extensión a lo continuo 
\section{Definición}

\begin{defi}
Una función es convergente a $l$ cuando $x\to a$ $\forall \varepsilon \in \mathbb{R}$ entonces $\exists \delta \geq 0 $ de manera que $|x-a|\leq \delta \Rightarrow |f(x)-f(a)|\leq 0$
\end{defi}

Podemos definir de esta manera el límite ya que hemos definido la estructura de espacio métrico de $\mathbb{R}$ ya que como demostramos anteriormente esta aplicación es una distancia. 

\begin{ejercicio}
Probar que la función $f(x)=x^2+x+1$ tiende a 1 en el punto $x=0$
\end{ejercicio}
\begin{proof}
Tomemos un $\varepsilon < 0$ entonces tomando como $\delta < min\lbrace 1, \dfrac{\varepsilon}{2}\rbrace\Rightarrow \delta <1 \Rightarrow \delta^2 <\delta$ por tanto, tenemos que $|x|<\delta$  

\begin{equation*}
|f(x)-1|=|x^2+x|\leq |x^2|+|x|< \delta^2+\delta<2\delta < \varepsilon
\end{equation*}
\end{proof}

\section{Propiedades de los límites}

\noindent
Una función puede tener distintos límites dependiendo de que topología esté definida sobre el conjunto a estudiar pero hay que recordar que una vez fijada la topología, este es único.
\noindent
Ahora vamos a ver y a comprender ciertas propiedades que nos ayudarán en el cálculo efectivo de límites
\noindent


\section{Cálculo efectivo de límites}


\section{Indeterminaciones}

\chapter{Las funciones sobre $\mathbb{R}$}
\minitoc

\newpage

\section{Definiciones previas}

\begin{defi}
Una función $f(x)$ es continua en un punto a cuando $\displaystyle \lim_{x\to a} f(x)=f(a)$. Es otra forma de definir la continuidad y es equivalente a la definición que se puede dar de una aplicación continua que dábamos en el apartado de topología. 
\end{defi}

\begin{defi}
Diremos que una función es continua si lo es en todos sus puntos. 
\end{defi}

\paragraph*{Propiedades}
Se cumplen las siguientes propiedades asociadas a las funciones continuas:
\begin{itemize}
\item Dadas $f$ y $g$ dos funciones continuas entonces tendremos que $f+g$ también es una función
\item Dada $f$ una función continua y $k\in \mathbb{R}$ entonces tenemos que $k\cdot f$ es una función continua también 
\item Dadas dos funciones $f, g$ continuas tenemos que $f\cdot g$ también es continua
\item La composición de funciones continuas $f,g$ también es una función  continua. 
\end{itemize}

\begin{ejercicio}
Demostrar mediante las propiedades de los límites de linealidad y de conservación del producto.
\end{ejercicio}

\begin{ejercicio}
Comprueba si las siguientes funciones son continuas en $x=2$
\begin{equation*}
\begin{array}{rrrr}
a) f(x)=\dfrac{1}{x-2}& b)f(x)=\dfrac{3x-5}{x^2-4}& c)f(x)=\dfrac{x^2}{x^2+1}& d)f(x)=3x^2-\dfrac{2}{x}\\
\end{array}
\end{equation*}

Comprobar que una función es continua en un punto $x=a$ es únicamente comprobar que el $\displaystyle lim_{x\to a} f(x)=f(a)$ por tanto hay que tener en cuenta cuando no existe un límite



\end{ejercicio}






\chapter{Derivabilidad sobre $\mathbb{R}$}

\minitoc

\newpage


\section{Concepto de la derivada}
Para empezar, tenemos que refrescar un concepto de geometría análitica, \emph{la pendiente de una recta}
\begin{defi}
La pendiente de una recta en $\mathbb{R}^2$ \emph{(El plano real)} se define como la cantidad de unidades que avanza la $y$ por cada unidad que avanza la $x$. Es decir, definiendo el incremento de y como $y_1-y_0=\Delta y$ donde $y_1$ es la coordenada y del punto final y $y_0$ lo mismo pero del punto inicial. definimos de manera igual el $\Delta x$. Entonces definimos de manera matemática la fórmula de la pendiente como :
\end{defi}
\begin{equation*}
m=\dfrac{\Delta y}{\Delta x}
\end{equation*}

\noindent
Ahora bien, sea $f(x)$ una función de manera que $f:\mathbb{R}\longrightarrow \mathbb{R}$ de la cual queremos obtener la recta secante que pasa por unos determinados puntos $p_1=(x_1,y_1),p_2=(x_2,y_2)$. Entonces tendremos la siguiente gráfica:\\
\noindent
Tendremos entonces que la fórmula de la recta secante a la función que pasa por esos dos puntos $p_1,p_2$ es la siguiente:
\begin{equation*}
(y-f(x_1))=\dfrac{\Delta f(x)}{\Delta x}(x-x_1)
\end{equation*}
\begin{defi}
A la pendiente de la recta secante a la función $f(x)$ en los puntos $x_1,x_2$ se le conoce como \textbf{\emph{Tasa de Variación Media}}
\end{defi}

\noindent
Supongamos ahora que escribimos $x_1=x$ y $x_2=x+h$ donde $h\in \mathbb{R}$ entonces la ecuación anterior queda como:
\begin{equation*}
(y-f(x_1))=\dfrac{f(x_1+h)-f(x_1)}{h}(x-x_1)
\end{equation*}

\noindent
Si después de esto, si hacemos que la $h\to 0$ obtendremos la recta tangente de manera que la pendiente  $\displaystyle \lim_{h\rightarrow 0}\dfrac{f(x+h)-f(x)}{h}$.

\noindent
Es ese límite lo que definimos como \emph{Derivada de una función}.
\begin{defi}
Llamaremos derivada de $f(x)$ en el punto $a$ al límite
\begin{equation*}
\lim_{h\to 0}\dfrac{f(a+h)-f(a)}{h}
\end{equation*}

\end{defi}
\newpage
\section{Derivabilidad de una función}
\begin{defi}
Diremos que una función es derivable en $a$ si existe el límite $\displaystyle \lim_{h\rightarrow 0}\dfrac{f(x+h)-f(x)}{h}$.
\end{defi}
\begin{defi}
Diremos que una función es derivable si lo es en todos los puntos del dominio.
\end{defi}
\subsection{Estudio de la derivabilidad de una función}

\newpage
\section{Tabla de derivadas}

Para empezar hay que tener en cuenta estas derivadas de operaciones de funciones básicas, sumar y restar, producto y división, producto por un escalar y composición

\noindent
Sean $a \in \mathbb{R}$ 

\section{Algunas demostraciones de fórmulas de derivadas}
\newpage

\section{}
\chapter{Aplicaciones de la derivada}
\minitoc

\newpage
\section{Cálculo de mínimos y máximos}
\section{Cálculo de la curvatura de las funciones}
\section{Optimización de funciones}
\chapter{Representación de funciones}
\minitoc

\newpage
\section{Dominio}
Recordemos lo que era el dominio de una función de manera precisa
\begin{defi}
El dominio de una función o aplicación $f: X\longrightarrow Y$ es el subconjunto de puntos $x \in X$ para el cual existe $f(x)$. 
\end{defi}
\noindent
Para que lo entendamos de nuevo la función $f$ en el caso que nos ocupa es algo a lo que le entran números, la función hace una operación, \emph{(En el caso de $2x$ sería que lo multiplica por dos)} y te devuelve ese valor operado. 
\noindent
El dominio son los números que la función puede operar sin  romperse, es decir, sin hacer cosas raras que no se pueden hacer en los números reales como por ejemplo:
\begin{itemize}
\item Dividir entre 0: Por tanto si tenemos una fracción habrá que comprobar cuando el denominador se hace 0. (Plantear la ecuación)
\item Hacer la raíz de un número negativo: Para poder hacerlo tendríamos que extender nuestro campo a $\mathbb{C}$ lo cual ahora mismo se nos escapa de nuestro alcance. Se plantea la inecuación
\item Hacer el logaritmo de un número negativo 
\end{itemize}

\subsection*{Dominios de las funciones elementales}
\noindent
Ahora vamos a ir desmenuzando los tipos de funciones que conocemos y analizando su dominio:
\begin{itemize}
\item \textbf{Funciones Polinómicas} Son las funciones del tipo $a_n\cdot x^n+\ldots+ a_o$ en este caso, su dominio es todo $\mathbb{R}$.
\item \textbf{Funciones Racionales} Son las funciones del tipo $\displaystyle \dfrac{a_n\cdot x^n+\ldots+ a_0}{b_m\cdot x^m+\ldots+ b_0}$ en este caso, su dominio es todo $\mathbb{R}$, salvo los puntos en los que se anule el denominador, por o tanto hay que resolver la ecuación $b_m\cdot x^m+\ldots+ b_0=0$.
\item \textbf{Funciones Irracionales} Son las de tipo $f(x)=\sqrt[n]{g(x)}$ tenemos que el dominio de $f(x)$ es:
\begin{itemize}
\item El mismo que $g(x)$ si n es \emph{impar}
\item Si n es par el dominio de $f(x)$ son los x que cumplen que  $g(x)\geq 0$. 
\end{itemize}
\item \textbf{Funciones Exponenciales}
Son de la forma $f(x)=a^{g(x)}$ con $a>0$ y $a\neq 1$, su dominio es $\mathbb{R}$.
\item \textbf{Funciones Logarítmicas}
Son de la forma $f(x)=log_a g(x)$ $a>0$ y  su dominio son los $x$ tales que $g(x)>0$.
\item \textbf{Funciones Trigonométricas Circulares} 
Tanto $f(x)=cos(x)$ como $f(x)=sen(x)$ su dominio es $\mathbb{R}$ y de aquí se pueden considerar el resto de funciones trigonométricas. 
\end{itemize}
\section{Simetría y periodicidad}
\noindent
La simetría de una función se define de la siguiente manera:

\begin{defi}
Una función $f:A \longrightarrow \mathbb{R}$ es par si $\forall x \in A$ se cumple que $f(-x)=f(x)$. Es decir, es simétrica respecto del eje de coordenadas $OY$
\end{defi}
%%Aquí va una imagen
\begin{defi}
Una función $f:A \longrightarrow \mathbb{R}$ es impar si $\forall x \in A$ se cumple que $f(-x)=f(x)$. Es decir, es simétrica respecto del origen de coordenadas 
\end{defi}
%Aquí va una imagen%
\noindent
Estas cualidades lo que nos permiten es reducir el tamaño del conjunto de puntos a estudiar, en el caso de la simetría nos permite reducir a la parte positiva de los reales y la periodicidad a un solo periodo. 

\section{Continuidad}
\section{Corte con los ejes}
\section{Asíntotas}
\section{Monotonía}
\section{Curvatura}
\chapter{Integración sobre $\mathbb{R}$}
\minitoc
\part{Ejercicios de Análisis Matemático}

\chapter{Representación de funciones}
\minitoc

\section*{Introducción}
\addcontentsline{toc}{section}{Introducción}

En esta capítulo vamos a recopilar todo los conocimiento de análisis que hemos recopilado durante todos los temas anteriores
\section{Funciones polinómicas}

\begin{ejercicio}
$$
f(x)=\dfrac{2x}{1+x^2}$$
\end{ejercicio}

\begin{proof}
Hola
\end{proof}

\section{Funciones racionales}


\section{Funciones irracionales}
\section{Funciones exponenciales}
\section{Funciones logarítmicas}
\section{Funciones trigonométricas}
\part{Álgebra lineal}
\chapter{Espacios Vectoriales }
\chapter{Aplicaciones lineales}
\chapter{Matrices}
\chapter{Determinantes}
\chapter{Discusión de sistemas}
\input{Probabilidad.tex}
\end{document}