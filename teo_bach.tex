\documentclass[12pt,a4paper]{book}

\usepackage{fancyhdr}


\usepackage[spanish]{babel}

\usepackage[spanish]{minitoc}
\usepackage{minitoc}

\addto{\captionsspanish}{\renewcommand{\mtctitle}{\'Indice del cap\'itulo}}

\renewcommand{\mtctitle}{\'Indice}% as� tambi�n cambiamos la cabecera del los minitoc

\usepackage{graphicx}

\usepackage{hyperref} % url, label, citas, tableofcontents

\title{Teoria de Matemáticas de Bachillerato}
\author{Pedro Ángel Fraile Manzano} 
\date{\today}



\begin{document}


%%%%%%%%%%%%%%%

%\input{Portada_book}%Portada con entorno titlepage
%\input{book_cabecera}%Personalizamos la portada a mano

\maketitle%Aparece el t�tulo (\title{} \author{} \date{\today)}), es distinto a la portada


%%%Personalizando el �ndice general (toc)
\addtocontents{toc}{\hspace{-7.5mm} \textbf{Apartados del libro}}
\addtocontents{toc}{\hfill \textbf{P\'agina} \par}
\addtocontents{toc}{\vspace{-2mm} \hspace{-7.5mm} \hrule \par}


\dominitoc%para crear �ndices en cada cap�tulo. Requiere paquete {minitoc}

\renewcommand{\contentsname}{Contenidos Generales}%Nombre del �ndice general

\tableofcontents
%\adjustmtc{} %indice en la intro

\part{Prefacios, Repaso y otras consideraciones}



\chapter{Operaciones sobre los números reales}

\minitoc

 
\section*{Definición} 
\addcontentsline{toc}{section}{Introducción}%para a�adir al �ndice la secci�n no numerada 


Esta secci\'on es la primera, aunque no la enumeramos
 
\section{Primera Secci\'on}
Esta es la primera secci\'on (numerada) del primer cap\'itulo.



\section{Segunda Secci\'on}
La segunda 

\chapter{Un segundo cap\'itulo}
%
\minitoc% Crea mini-�ndice al principio del capitulo

\section{}
\section{}


\appendix
\chapter{Mi primer ap\'endice}

\end{document}