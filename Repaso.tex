\part{Prefacios, Repaso y otras consideraciones}



\chapter{Operaciones sobre los números reales}

\minitoc

\newpage
 
\section*{Introducción} 
\addcontentsline{toc}{section}{Introducción}%para a�adir al �ndice la secci�n no numerada 

Los distintos conjuntos de números surgen de la necesidad de resolver distintas ecuaciones, es decir, a medida que necesitamos resolver ecuaciones más complejas, más se amplían el campo de números con los que podemos actuar:
 
\section{Estructura de los números reales}
Los números reales tiene estructura de cuerpo y te preguntarás ¿ Qué es un cuerpo?
\begin{defi}
Un cuerpo es una terna $(\mathbb{K},+,\cdot)$ donde:
\begin{enumerate}
\item $\mathbb{K}$ es un conjunto de elementos 
\item $+$ es una operación sobre los elementos de $\mathbb{K}$ que cumple:
\begin{itemize}
\item Es una operación \textbf{conmutativa}, es decir, sean $a,b\in \mathbb{K}$ entonces tendremos que $a+b=b+a$
\item Es una operación \textbf{asociativa}, es decir dados $a,b,c \in \mathbb{K}$ tenemos que $a+(b+c)=(a+b)+c$
\item Existe un \textbf{elemento neutro}, es decir $\exists e / e+a=a+e=a$ $\forall a \in \mathbb{K}$.
\item Cada elemento $a \in \mathbb{K} $ existe un elemento \textbf{inverso} que se denota por $a^{-1} $ de tal manera que $a+a^{-1}=a^{-1}+a=e$ (\emph{Esto también se da cuando no se cumple la conmutativa})
\end{itemize}

\item $\cdot$ es una operación que cumple lo siguiente 
\begin{itemize}

\item Es una operación \textbf{asociativa}, es decir dados $a,b,c \in \mathbb{K}$ tenemos que $a\cdot(b\cdot c)=(a\cdot b)\cdot c$

\item Existe un \textbf{elemento neutro} para esta operación  $\exists e / e\cdot a=a \cdot e=a$ $\forall a \in \mathbb{K}$.

\item Para todo elemento $a \in \mathbb{K}$ entonces $\exists a^{-1} /  a \cdot a^{-1}=a^{-1}\cdot a=e $ (\emph{Esto es lo que distingue un cuerpo a un anillo})

\item $\cdot$ es \textbf{distributivo} respecto de $+$ es decir, $a\cdot (b+c)=a \cdot b + a \cdot b$

\end{itemize}
\end{enumerate}

  
\end{defi}
\newpage
\noindent
\textbf{\emph{Aclaración 1: }} Aunque se denoten como $+, \cdot$ no tenemos por qué usar las definiciones habituales de la suma y la multiplicación. Por ejemplo, la suma y producto de números reales no son iguales que las mismas operaciones para las matrices \emph{(quedaros con ese nombre.) } \\
\textbf{\emph{Aclaración 2: }}  De esta manera que tenemos que lo que llamamos en los números reales la resta es la suma por el inverso y la división es el producto por el inverso.

\begin{ejerci}

Demostrar que $\mathbb{R}$ y $\mathbb{C}$ son cuerpos 

\end{ejerci}



\section{Potencias y Logaritmos}
\begin{defi}
Podemos definir las potencias como  $a^n=\overbrace{a \cdot\ldots \cdot a}^{\text{n veces}}$. Una vez entendido esto tenemos las siguientes propiedades
\end{defi}

\paragraph{Propiedades}
\begin{enumerate}
\item $a^1=a$ y $a^0=1$ para cualquier $a\in\mathbb{R}$
\item $a^{-1}=\dfrac{1}{a}$
\item $a^n\cdot a^m=a^{n+m}$
\item $\dfrac{a^n}{a^m}=a^{n-m} $
\item $\left( a^n \right) ^m =a^{n \cdot m}$
\item $\sqrt[n]{a}=a^{\frac{1}{n}}$
\item $(a \cdot b)^n=a^n \cdot b^n$
\item $ \left(\dfrac{a}{b}\right)^n=\dfrac{a^n}{b^n}$
\end{enumerate}
\paragraph{Demostración}
\begin{enumerate}
\item Para la primera demostración no hace falta más que decir que estamos ``poniendo'' sólo una a y que $a^0=1$ es básicamente proveniente del álgebra $\mathbb{Z}$ modular.
 
\item En este caso,  tenemos que al utilizar la propiedad 3 quedará más clara pero si nosotros tenemos $a^1 \cdot a^{-1}=a^0=1 \Rightarrow a^{-1}=\dfrac{1}{a}$

\item Ahora tenemos que $a^n\cdot a^m=\overbrace{a \cdot\ldots \cdot a}^{\text{n veces}}\cdot \overbrace{a \cdot\ldots \cdot a}^{\text{m veces}}=\overbrace{a \cdot\ldots \cdot a}^{\text{n+m veces}}=a^{m+n}$
\item Si combinamos la propiedad 2 y 3 queda probado $\dfrac{a^n}{a^m}=a^n\cdot \dfrac{1}{a^m}=a^n \cdot a^{-m}=a^{n-m}$
\item Este se debe a que estamos multiplicando paquetitos del producto de n a's, es decir ,
$ \left( a^n \right) ^m =\overbrace{a^n \cdot \ldots \cdot a^n}^{\text{m veces}}=\overbrace{\underbrace{a \cdot \ldots \cdot a}_{\text{n veces}}\cdot \ldots \cdot \underbrace{a \cdot\ldots \cdot a}_{\text{n veces}}}^{\text{m veces}}=\overbrace{a \cdot\ldots \cdot a}^{\text{n m veces}}=a^{m n}$
\item Haciendo un razonamiento análogo pero con el producto lo tenemos 
\item Tenemos lo siguiente $(a \cdot b)^n=\overbrace{a\cdot b \cdot\ldots \cdot a\cdot b}^{\text{n veces}}=\overbrace{a \cdot\ldots \cdot a}^{\text{n veces}}\cdot \overbrace{b \cdot\ldots \cdot b}^{\text{n veces}}=a^n \cdot b^n$
\item Utilizando un razonamiento similar al anterior lo tenemos cambiando únicamente $b$ por $b^{-1}$
\end{enumerate}

\begin{defi}
Definimos el logaritmo de $b \in \mathbb{R}^+$ en base $a>0$ de la siguiente manera
\begin{equation}
log_a b=x \Leftrightarrow a^x=b
\end{equation}
\end{defi}

Esta definición nos permite "traducir" de logaritmos a potencias y es lo que se utiliza para demostrar las siguientes propiedades
%%@todo Lo de propiedades hay que hacer un entorno no numerado con el estilo de definition y lo mismo para las demostraciones 
\paragraph{Propiedades: }
Sean $P,Q,a \in \mathbb{R}^+ $
\begin{enumerate}
\item $log_a 1=0$
\item $log_a a=1$
\item $log_a (P\cdot Q)=log_a P +log_a Q$
\item $log_a \left(\dfrac{P}{Q}\right)=log_a P -log_a Q$
\item $log_a P^n=n\cdot log_a P$
\end{enumerate}

\begin{ejerci}
Se propone al lector la demostración de estas propiedades utilizando la definición de logaritmos y las propiedades de las potencias.
\end{ejerci}
\newpage
\section{Resolución de ecuaciones exponenciales}

\begin{defi}
Podemos definir una ecuación exponencial como aquella que tiene la incógnita en el exponente
\begin{equation*}
a^x=b
\end{equation*}
\end{defi}
\noindent
Podemos distinguir los siguientes casos:
\begin{itemize}
\item \textbf{\textit{Ecuaciones donde la incógnita aparece en un solo exponente}}

El procedimiento es intentar poner todos los elementos como potencias de la base que tiene la incógnita
\begin{align*}
2^{x+1}&=8\\
2^{x+1}&=2^3
\end{align*}
Tras esto, podemos hacer el logaritmo de cada uno de los lados ya que $log_a P=log_a Q \Leftrightarrow P=Q$ en este caso $a=2$ de tal forma que lo anterior nos queda:
\begin{align*}
2^{x+1}&=2^3\\
log_2( 2^{x+1})&=log_2 (2^3)\\
x+1&=3\\
x&=2
\end{align*}
También puede que no podamos descomponer en potencias de una sola base entonces tenemos el siguiente caso.
\begin{equation*}
2^x=127
\end{equation*}
Entonces tomamos logaritmos para poder resolverlo
\begin{align*}
2^x&=127\\
log_2(2^x)&=log_2(127)\\
x \cdot log_2(2)&=log_2(127)\\
x&=log_2(127)
\end{align*}
A partir de aquí podemos utilizar un cambio de base de los logaritmos para poder usar el logaritmo en base $10$ o $e$.

\begin{ejercicio}
Resuelve las siguientes ecuaciones. 
\\
\begin{tabular}{r c}
a) $4^{x+1}-8=0$& b) $3^{x+2}=81$& $$
\end{tabular}
\end{ejercicio}
\newpage 
\item \textbf{\textit{Ecuaciones donde la incógnita está en más de una potencia}}
El procedimiento es conseguir una expresión donde las potencias que tengan las incógnitas se reduzcan a las misma base y podamos hacer un cambio de variable $a^x=t$ que después desharemos como si fuera un caso como el anterior. 

\end{itemize}


\newpage
\section{Resolución de ecuaciones logarítmicas}
\begin{defi}
Podemos definir una ecuación exponencial como aquella que tiene la incógnita dentro de un logaritmo.
\end{defi}



\chapter{Polinomios sobre el cuerpo de los reales y ecuaciones}

\minitoc

\section{Conceptos básicos}
\section{Operaciones con polinomios}
\section{Divisibilidad de polinomios}

\chapter{Ecuaciones polinómicas}
\minitoc

\section{Ecuaciones lineales}
\section{Ecuaciones parabólicas}
\section{Ecuaciones de grado mayor que 2}

\chapter{Inecuaciones}
\minitoc

\section{Inecuaciones lineales}
\section{Inecuaciones no lineales}


